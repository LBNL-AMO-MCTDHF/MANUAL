\begin{verbatim}
!! PARAMETERS FILE FOR SINC DVR POLYATOMIC BASIS AND HAMILTONIAN:
!!            sincparinp namelist input.

module myparams
implicit none

!! FOR TOTAL ORBITAL PARALLELIZATION, SET orbparflag=.true., AND
!! parorbsplit=3 in &parinp
logical :: orbparflag=.false.

!! THE FOLLOWING FLAG IS THEN RELEVANT.  Option for parallel KE matvec,
!!   rate limiting step in algorithm.
integer :: zke_paropt=1   !! 0=sendrecv 1=SUMMA (bcast before) 2=reduce after

!! toepflag:  Fast Fourier transforms for Toeplitz matrix vector multiplications
!! toepflag=1, triple toeplitz T^-1.  toepflag=2, T^-1 and single toeplitz T. 
integer :: toepflag=1

!! fft_batchdim: determines batch size for matrix elements and 
!! fft_circbatchdim: determines sub batch size for FFT 
!!    defaults set small (less memory, more MPI messages) to avoid MPI problems when doing large 
!!    calculations.  Otherwise bigger values will be faster.  There is a message size sweet spot
!!    on many machines.
integer :: fft_batchdim=1     !! 1 = do nspf matrix elements in nspf batches (less memory)
                              !! 2 = do nspf^2 in one batch (faster unless MPI problems)
integer :: fft_circbatchdim=1 !! 0,1,2, circbatchdim < batchdim; larger faster unless MPI problems
integer :: fft_mpi_inplaceflag=1     !! fft_mpi_inplaceflag:
                                     !!  0 = out-of-place fft, out-of-place fft inverse
                                     !!      3d FFT + (summa/circ C.T. depending on fft_ctflag)
                                     !!  1 = 3 x (1d FFT , all-to-all index transposition)
integer :: fft_ct_paropt=1           !! fft_ct_paropt, relevant if fft_mpi_inplaceflag=0
                                     !!    like zke_paropt: 0 = sendrecv 1 = summa
integer :: fft_mpi_keinplace=1       !! for toepflag=2 (single toeplitz matvec for KE) equals 
                                     !!    fft_mpi_inplaceflag by default; specify in namelist
                                     !!    if you want FFT methods to be different for T and T^-1
integer :: num_skip_orbs=0
integer :: orb_skip(200)=-1

integer :: toothnsmall=40
integer :: toothnbig=240

integer :: numcenters=1
integer :: centershift(3,100)=0    !! grid point index for each center
real*8 :: nuccharges(100)=2d0

integer :: numpoints(100)=15
real*8 :: spacing=0.25d0

integer :: orblanorder=500         !! krylov order for block lanczos orbital calculation
integer :: orblancheckmod=10       !! check every
real*8 :: orblanthresh=1d-4

integer :: capflag=0               !! Number of complex absorbing potentials
integer :: capmode=0               !! Capmode=1 is 
integer :: cappower(100)=2         !!   v_i(r)= capstrength_i*(r/capstart_i)^cappower_i
real*8 :: capstart(100)=0.001d0    !! Capmode=0 is    
real*8 :: capstrength(100)=0.01d0  !!   v_i(r)= capstrength_i*max(0,r-capstart_i)^cappower_i
real*8 :: mincap=0d0 , maxcap=1d30 !! V_CAP = -i* max(mincap,min(maxcap,sum_i v_i))
\end{verbatim}
