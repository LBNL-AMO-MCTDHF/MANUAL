\begin{verbatim}
!! *********************************************************************************************************** !!
!!   Parameters for MCTDHF calculation; parinp NAMELIST input from Input.Inp (default)
!! *********************************************************************************************************** !!
!!
!! Type, variable, default value !! Command-line !! Description 
!!                               !!  option      !! 
\end{verbatim}
!!{\large \quad HAMILTONIAN PARAMETERS}
\begin{verbatim}
module ham_parameters
integer :: nonuc_checkflag=1     !!              !! Turn off deriv operators in nuclear dofs.
integer :: tdflag=0              !! Pulse        !! Use pulse?
integer :: velflag=0             !!              !!  Length (V(t)) or velocity (A(t))      
!!  Constraintflag=1: Density matrix constraint: assume nothing, keep constant off block diag
!!  2: Dirac-Frenkel (McLachlan/Lagrangian) variational principle.
integer :: constraintflag=0      !! Constraint=  !! As described below (see CONSTRAINT)
integer :: denmatfciflag=0                       !! If .ne. 0 then does denmat constrant as programmed
                                                 !!  before Miyagi's help
DATATYPE :: energyshift=0d0      !!              !! complex shift for making energy real for imperfect CAP/ECS
integer :: drivingflag=0                         !!  Solve for the change in the wave function not wave function 
real*8 :: drivingproportion=0.999999999999d0     !!   -- "psi-prime" treatment.
DATATYPE :: timefac=&            !! Prop/        !! d/dt psi = timefac * H * psi
        DATANEGONE               !!  Relax       !!
real*8 :: mshift=0d0                             !! shift configurations based on m-value.. to break 
                                                 !!  degeneracy for state averaged sym restricted
                                                 !!  (mrestrictmin, mrestrictmax) mcscf; good idea.
end module ham_parameters
\end{verbatim}
!!{\large \quad PULSE.  (If tdflag=1) }
\begin{verbatim}
module pulse_parameters
integer :: numpulses=1
integer ::  pulsetype(100)=1      !!              !!  Pulsetype=1:  A(t) = pulsestrength * sin(w t)^2,
real*8  :: omega(100)=1.d0        !!              !!  2:  A(t) = strength * sin(w t)^2 
real*8 :: omega2(100)=1.d0        !!              !!             * sin(w2 t + phaseshift),
real*8 :: pulsestart(100)=0d0     !!              !!   
real*8 :: phaseshift(100)=0d0     !!              !!    pulsestart < t < pulsestart + pi/w; 0 otherwise
real*8 :: chirp(100)=0d0          !!              !!
real*8 :: ramp(100)=0d0
real*8 :: longstep(100)=1d0       !!              !!  Pulsetype 3 available: monochromatic, sinesq start+end
!! NOW COMPLEX
DATATYPE :: pulsestrength(100)=.5d0 !!            !!  A_0 = E_0/omega (strength of field)  
real*8 :: intensity(100)= -1.d0   !!              !! overrides pulse strength.  Intensity, 10^16 W cm^-2 
real*8 :: pulsetheta(100)=0.d0    !!              !!  angle between polarization and bond axis (radians)
real*8 :: pulsephi(100)=0.d0      !!              !!  polarization in xy plane
real*8 :: maxpulsetime=1.d20     !!              !!  
real*8 :: minpulsetime=0.d0      !!              !!  By default calc stops after pulse (overrides finaltime,
                                                 !!   numpropsteps); this will enforce minimum duration
integer :: conjgpropflag=0       !! for complex Domcke
end module pulse_parameters
\end{verbatim}
!!{\large \quad CONFIGURATIONS / SLATER DETERMINANTS}
\begin{verbatim}
module df_parameters
integer :: df_restrictflag=0      !!              !! apply constraint to configuration list?  Must use this
                                                 !!  option if constraintflag /= 0.  1 is sufficient;
                                                 !!  dfrestrictflag=2 necessary for action 22. 
                                                 !!  SEE MANUAL FOR PROPER USE OF dfrestrictflag/shell options.
end module df_parameters
module basis_parameters
integer :: holeflag=0            !! Holes        !! index holes not electrons
integer :: mrestrictflag=0       !!              !! If spfrestrictflag=1, restrict wfn to given total M.
integer :: mrestrictval=0        !!              !!    This is the value.
integer :: mrestrictmax= 99999   !!              !! If doing state averaged MCSCF, can include a range of m vals;
integer :: mrestrictmin=-99999   !!              !!    set these variables, with mrestrictflag=0, spfrestrictflag=1
integer :: ugrestrictflag=0      !!              !! like mrestrictflag but for parity
integer :: ugrestrictval=1       !!              !!    like mrestrictval but for parity (1=even,-1=odd)
integer :: restrictflag=1        !!              !! Restrict spin projection of determinants?
\end{verbatim}
!! \textbf{\qquad For restricted configuration lists (not full CI) (see manual)}
\begin{verbatim}
integer :: numshells=1           !!              !! number of shells.  greater than one: possibly not full CI. 
!!$ integer :: shelltop(100)=-1  !! Numfrozen=   !! shelltop is namelist input in parinp; the internal variable 
                                                 !!    is allshelltop.   shelltop(1) only may be assigned via 
                                                 !!    command line, with Numfrozen.
integer :: numexcite(100)=999    !! Numexcite=   !! excitations from core shells (i.e. defined for shells 1 
                                                 !!    through numshells-1).  Only numexcite(1) may be 
                                                 !!    assigned via command line input.
integer :: minocc(100)=-999      !!              !! minimum occupation, each shell
integer :: maxocc(100)=999       !!              !!    maximum
integer :: shells(100)=1
integer :: allshelltop(0:100)=0           
\end{verbatim}
!!{\large \quad ORBITALS (SINGLE PARTICLE FUNCTIONS, SPFS)}
\begin{verbatim}
integer :: spfrestrictflag=0     !!              !! Restrict m values of orbitals?  
integer :: spfmvals(1000)=0      !!              !!   M-values of orbitals 
integer :: spfugrestrict=0       !!              !! Restrict parity of orbitals? 
integer :: spfugvals(1000)=0     !!              !!   Parity (+/-1; 0=either) of orbitals (ungerade/gerade)
end module basis_parameters
\end{verbatim}
!!{\large \quad for SPARSE CONFIGURATION ROUTINES - if sparseconfigflag .ne. 0}
\begin{verbatim}
module lanparameters
integer :: lanprintflag=0
integer :: lanczosorder=200      !!              !!   lanczos order used in A-vector eigen.
integer :: lancheckstep=20       !!              !! lanczos eigen routine checks for convergence every this # steps
real*8 :: lanthresh=1.d-9        !!              !! convergence criterion.
end module lanparameters
module sparse_parameters
integer :: sparseconfigflag=0    !! Sparse       !! Sparse configuration routines on or off (for large # configs)
integer :: sparseopt =1                          !! 0= direct CI  1= sparse matrix algebra (faster, more memory)
integer :: par_consplit=1                        !! Fully distribute a-vector
integer :: sparsesummaflag=0                     !! a-vector matvec mode 0=gather 1=summa(bad) 2=sendrecv(ok)
integer :: sparsedfflag=1                        !! If zero disable separate restricted config walk list
integer :: nonsparsepropmode=1                   !! 0 = ZGCHBV expokit; 1 = mine expmat
integer :: nzflag=1                              !! use only processors with nonzero number of Slaters for a-vec
logical :: shuffle_dfwalktype=.true.             !! redistribute restricted configs evenly over processors
integer :: sparseprime=1                         !! For reordering config list (experimental)
end module sparse_parameters
\end{verbatim}
!!{\large \quad Orbital parallelization}
\begin{verbatim}
module spfsize_parameters
integer :: parorbsplit=1                !! 1=keep all orbs, prop in parallel   3=polyatomic, full MPI
end module spfsize_parameters
\end{verbatim}
!!{\large \quad CONSTRAINT: With constraintflag. NEED FOR RESTRICTED CONFIG LIST.}
\begin{verbatim}
module constraint_parameters
real*8 :: lioreg= 1d-9           !!              !! Regularization for linear solve constraint
integer :: conway=0                              !! for constraintflag=2, dirac frenkel constraint
                                                 !!   0=McLachlan 1=50/50 mix 2=Lagrangian
                                                 !!   3=Lagrangian with epsilon times McLachlan
real*8 :: conprop=1d-1                           !! epsilon for conway=3
real*8 :: condamp=1
end module constraint_parameters
\end{verbatim}
!!{\large \quad Timing}
\begin{verbatim}
module timing_parameters
integer :: notiming=2            !!NoTiming=0,1,2!! 0=write all 1=write some 2= write none
                                 !!  Timing=2,1,0!!     controls writing of all timing and some info files
integer :: timingout=499         !!              !! various routines output to file (timing info) every this 
                                 !!              !!   # of calls
character(len=200):: timingdir="timing"                       !!
end module timing_parameters
\end{verbatim}
!!{\large \quad Biorthogonalization }
\begin{verbatim}
module bio_parameters
integer ::      maxbiodim=100, & !! Max krylov dim for biorthogonalization
     biodim=10                   !! Starting krylov dim for biorthogonalization
real*8 ::     biotol=1.d-6       !! Expokit tolerance parameter for biorthogonalization
integer :: logbranch=1           !! branch of logarithm 3 options 0,1,2
end module bio_parameters
\end{verbatim}
!!{\large \quad Regularization}
\begin{verbatim}
module tol_parameters
real*8 :: lntol=1d-4             !! Regularization of natural logarithm
real*8 :: invtol=1d-8            !! Regularization of inverses especially for spectral expansions
end module tol_parameters
module denreg_parameters
real*8 :: denreg=1d-10           !! Denreg=      !! density matrix regularization parameter.
end module denreg_parameters
module parameters
  use littleparmod;  use fileptrmod;  use r_parameters; use sparse_parameters; use tol_parameters
  use ham_parameters;  use basis_parameters;  use timing_parameters; use spfsize_parameters;
  use df_parameters; use dotmod;    use constant_parameters
  implicit none
\end{verbatim}
!!{\large \quad MAIN PARAMETERS }
\begin{verbatim}
integer :: mcscfnum=1            !! MCSCF=       !! Number of A-vectors (state avgd mcscf or prop)
integer :: numelec=2             !!              !! NUMBER OF ELECTRONS
integer :: orbcompact=1          !!              !! Compact orbitals for expo prop with spfrestrictflag?  Probably ok.
integer :: saveflag=1                            !! if zero does not save wave function at the end
integer :: save_every=0                          !! if nonzero saves wave function every save_every mean field steps
integer :: walkwriteflag=0                       !! Turning OFF writing of walks by default
integer :: spf_flag=1            !!              !! IF ZERO, FREEZE SPFS. (for debugging, or TDCI)
integer :: avector_flag=1        !!              !! IF ZERO, FREEZE AVECTOR. (for debugging)
character (len=200) :: &         !!              !! MAY BE SET BY COMMAND LINE OPTION ONLY: not namelist
  inpfile="Input.Inp        "    !! Inp=filename !!  input.  (=name of input file where namelist input is)
\end{verbatim}
!!{\large \quad PROPAGATION/RELAXATION}
\begin{verbatim}
real*8 :: par_timestep=0.1d0     !! Step=        !! MEAN FIELD TIMESTEP
integer :: prepropflag=0         !!              !! Option for three-step propagation for strong fields or constraint
integer :: improvedrelaxflag=0   !! Relax        !! For improved versus regular relaxtion.   
integer :: threshflag=0          !!              !! Set to 1 for regular relaxation
real*8 :: expotol=1d-8           !!              !! Orbital krylov convergence parameter
integer :: expodim=10            !!              !! Starting krylov dimension for expokit orbital propagation
integer :: maxexpodim=100        !!              !! Maximum krylov dimension for expokit orbital propagation
real*8  :: expostepfac=1.2d0     !!              !! Miscellaneous algorithm parameter
\end{verbatim}
!!\textbf{\qquad SPARSE - if sparseconfigflag .ne. 0}
\begin{verbatim}
real*8 :: aerror=1d-9            !!              !! error criterion for sparse a-vector prop
                                                 !! and eigenfunctions with improvedquad=1,3
integer :: maxaorder=100         !!              !! max lanczos order for sparse a-vector expokit prop or
                                                 !!    dgmres solve for avector improvedquadflag=1,3
\end{verbatim}
!!{\large \quad PROPAGATION}
\begin{verbatim}
integer :: littlesteps=1         !!              !! Sub intervals of mean field time step for avector prop
real*8 :: finaltime=4d4          !! T=           !! length of prop.  Overridden for pulse and relax.  
\end{verbatim}
!!\textbf{\qquad SPARSE - if sparseconfigflag .ne. 0}
\begin{verbatim}
integer :: aorder=30             !!              !! Starting lanczos order for sparse a-vector expokit prop
\end{verbatim}
!!{\large \quad RELAXATION}
\begin{verbatim}
integer :: improvednatflag=0     !!              !! If improved relax, replace with natorbs every iteration
real*8 :: stopthresh=1d-5        !!              !! Spf error tolerance for relaxation convergence (PRIMARY)
real*8 :: astoptol=1d-7                          !! Avector error tolerance for relax  (BACKUP - WAS STOPTHRESH)
real*8 :: timestepfac=1d0                        !! accelerate relax. multiply par_timestep by this each time
real*8 :: max_timestep=1d10                      !!    maximum time step (limit on exponential growth)
integer :: improvedquadflag=0    !!              !! Use newton iteration not diagonalization for improvedrelax.
                                                 !!     (1 = A-vector, 2 = orbitals, 3 = both)
real*8 :: quadstarttime=-1d0                     !! Waits to turn on orbital quad (2 or 3) until this time
real*8 :: aquadstarttime=-1d0                    !! Waits to turn on avector quad (1 or 3) until this time
real*8 :: maxquadnorm=1d10                       !! brakes to use if improvedquadflag=2 or 3 is diverging
real*8 :: quadtol=1d-1           !!              !! Threshold for solution of Newton solve iterations orbitals.
integer :: quadprecon=1          !!              !! Precondition newton iterations for A-vector?
integer :: quadorthflag=0        !!              !! If eigenfunctions are becoming linearly dependent try this
integer :: normboflag=0          !!              !! Enforce norm at each r value
\end{verbatim}
!!\textbf{\qquad SPARSE - for relaxation if sparseconfigflag .ne. 0 see module lanparameters above, and}
\begin{verbatim}
integer :: maxdgdim = 800                        !! Max dimension for improvedquadflag options
\end{verbatim}
!!{\large \quad ORBITALS (SINGLE PARTICLE FUNCTIONS, SPFS)}
\begin{verbatim}
integer :: nspf=1                !! Nspf=        !! number of orbitals
integer :: numfrozen=0           !!              !! number of doubly occ orbs (removed from calculation)
\end{verbatim}
!!{\large \quad CONFIGURATIONS}
\begin{verbatim}
integer :: restrict_ms=0         !!              !! For restrictflag=1: 2*m_s: 2x total m_s (multiplicity of 
                                                 !!    lowest included spin states minus one)
integer :: spin_restrictval=0    !!              !! For allspinproject=1: determines spin. Default high spin S=M_s.
                                                 !!  To override use this variable. Equals 2S if S^2 eigval is S(S+1)
integer :: all_spinproject=1     !!              !! Constrain S(S+1) for propagation?
\end{verbatim}
!!{\large \quad INITIALIZATION}
\begin{verbatim}
integer :: loadavectorflag=0     !! A=file       !! load avector to start calculation?
integer :: numavectorfiles=1     !!              !! number of avector files containing a-vectors to load into the
                                                 !! mcscfnum available slots, or if load_avector_product.ne.0,
                                                 !! number of one-wfn files for product, w/ total numelec=numelec
integer :: load_avector_product=0                !! make product wave function for multiple-molecule load
integer :: loadspfflag=0         !! Spf=file     !! load spfs to start calculation?  
                                                 !!    (Otherwise, core eigenfunctions.)
integer :: reinterp_orbflag=0                    !! sinc dvr only, half spacing interpolation for orb load
integer :: spf_gridshift(3,100)=0                !! sinc dvr only, shift orbitals on read, slow index spffile
integer :: numspffiles=1         !!              !! for multiple-molecule (e.g. chemistry) calcs, load many
integer :: numskiporbs=0         !!              !! Reading orbs on file(s), skips members of combined set.
integer :: orbskip(1000)=0       !!              !! Which to skip
character (len=200) :: &         !! A=file       !! A-vector binary file to read.  Can have different configs
     avectorfile(MXF)="Bin/avector.bin"          !!   but should have same number of electrons.   
character (len=200) :: &         !! Spf=file     !! Spf file to read.  Can have fewer m vals, smaller radial 
     spffile(MXF)="Bin/spfs.bin"                 !!   grid, or fewer than nspf total orbitals. 
integer :: avecloadskip(100)=0
integer :: numholes=0                            !! Load a-vector with this many more electrons and annihilate
integer :: numholecombo=1                        !! Number of (products of) annihilation operators to combine 
                                                 !!   (for spin adapt)
integer :: numloadfrozen=0                       !! For loading a vector with orbitals to be frozen (dangerous)
integer, allocatable :: myavectorhole(:,:,:)     !! Namelist input is avectorhole.  Fast index numholes (#
                                                 !!   annihilation operators to multiply together); then    
                                                 !!   numholecombo, number of such products to combine; then
                                                 !!   mcscfnum, wfn of current propagation.
integer :: excitations=0                         !! Similar to holes: number of products of excitation ops
integer :: excitecombos=1                        !!   number of products to linearly combine
integer,allocatable:: myavectorexcitefrom(:,:,:) !! Similar to avectorhole.  Namelist input avectorexcitefrom, etc.
integer,allocatable:: myavectorexciteto(:,:,:)   !!
                                                 !! For both excite and hole: value is spin orbital index
                                                 !! 1=1alpha, 2=1beta, 3=1alpha, etc.
                                                 !! negative input -> negative coefficent
integer :: messflag=0                            !! add static to initial orbitals
real*8 :: messamount=1.d-4                       !!    how much
integer :: messavec=0                            !! add static to initial a-vector
real*8 :: messaamount=1.d-4                      !!    how much
\end{verbatim}
!!{\large \quad INPUT / OUTPUT }
\begin{verbatim}
character(len=200):: finalstatsfile="Dat/finalstats.dat"         !! output for relaxation calculation
real*8 :: pulseft_estep=0.01d0                                   !! step in hartree for Pulseft.Dat output files
character (len=200) ::      avectoroutfile="Bin/avector.bin"     !! A-vector output file.        
character (len=200) ::      spfoutfile="Bin/spfs.bin"            !! Spf output file.
character(len=200):: psistatsfile="Dat/psistats.dat"             !! for action 25
integer :: psistatfreq=1                                         !!  "
character(len=200):: dendatfile="Dat/denmat.eigs.dat"            !! if notiming=0
character(len=200):: denrotfile="Dat/denmat.rotate.dat"          !!  "
character(len=200):: rdendatfile="Dat/rdenmat.eigs.dat"          !! deprecated
character (len=200) :: ovlspffiles(50)="Bin/ovl.spfs.bin"        !! for actions 20 and 26 
character (len=200) :: ovlavectorfiles(50)="Bin/ovl.avector.bin" !!      (see numovlfiles in ACTIONS)
character(len=200):: outovl="Dat/Overlaps.dat"                   !! for action 20
character(len=200):: outmatel="Dat/Matel.dat"                    !! for action 20
integer :: act21circ=0                                           !! set nonzero to enable circular polarization 
                                                                 !!     output for action 21
character(len=200):: xdipfile="Dat/XDipoleexpect.Dat",&          !! for action 21 dipole moments D(t)
 ydipfile="Dat/YDipoleexpect.Dat",zdipfile="Dat/ZDipoleexpect.Dat",&    !! "
 xydipfile="Dat/XYDipoleexpect.Dat",xzdipfile="Dat/XZDipoleexpect.Dat",&!! "  circular polarization
 yxdipfile="Dat/YXDipoleexpect.Dat",yzdipfile="Dat/YZDipoleexpect.Dat",&!! "   XY = X->Y, Y->(-X) ; etc.
 zxdipfile="Dat/ZXDipoleexpect.Dat",zydipfile="Dat/ZYDipoleexpect.Dat"  !! "
character(len=200):: xdftfile="Dat/XDipoleft.Dat",&              !! for action 21 dipole moments D(omega)
 ydftfile="Dat/YDipoleft.Dat", zdftfile="Dat/ZDipoleft.Dat",&    !! and emission/absorption
 xydftfile="Dat/XYDipoleft.Dat",xzdftfile="Dat/XZDipoleft.Dat",& !!  "  circ polarization
 yxdftfile="Dat/YXDipoleft.Dat",yzdftfile="Dat/YZDipoleft.Dat",& !!  "
 zxdftfile="Dat/ZXDipoleft.Dat",zydftfile="Dat/ZYDipoleft.Dat"   !!  "
character(len=200):: xoworkfile="Dat/XOWork.Dat",&               !! for action 21 work per pulse integral domega
 yoworkfile="Dat/YOWork.Dat",zoworkfile="Dat/ZOWork.Dat",&       !!  "
 xyoworkfile="Dat/XYOWork.Dat",xzoworkfile="Dat/XZOWork.Dat",&   !!  "  circ polarization
 yxoworkfile="Dat/YXOWork.Dat",yzoworkfile="Dat/YZOWork.Dat",&   !!  "
 zxoworkfile="Dat/ZXOWork.Dat",zyoworkfile="Dat/ZYOWork.Dat"     !!  "
character(len=200):: xtworkfile="Dat/XTWork.Dat",&               !! for action 21 work per pulse integral dt
 ytworkfile="Dat/YTWork.Dat",ztworkfile="Dat/ZTWork.Dat",&       !!  "
 xytworkfile="Dat/XYTWork.Dat",xztworkfile="Dat/XZTWork.Dat",&   !!  "  circ polarization
 yxtworkfile="Dat/YXTWork.Dat",yztworkfile="Dat/YZTWork.Dat",&   !!  "
 zxtworkfile="Dat/ZXTWork.Dat",zytworkfile="Dat/ZYTWork.Dat"     !!  
character(len=200):: xophotonfile="Dat/XOPhoton.Dat",&                 !! for action 21 work per pulse integral domega
 yophotonfile="Dat/YOPhoton.Dat",zophotonfile="Dat/ZOPhoton.Dat",&     !!  "
 xyophotonfile="Dat/XYOPhoton.Dat",xzophotonfile="Dat/XZOPhoton.Dat",& !!  "  circ polarization
 yxophotonfile="Dat/YXOPhoton.Dat",yzophotonfile="Dat/YZOPhoton.Dat",& !!  "
 zxophotonfile="Dat/ZXOPhoton.Dat",zyophotonfile="Dat/ZYOPhoton.Dat"   !!  "
character(len=200):: corrdatfile="Dat/Correlation.Dat"        !! for action 1
character(len=200):: corrftfile="Dat/Corrft.Dat"              !!  "
character(len=200):: fluxmofile="Flux/flux.mo.bin"            !! for actions 15,16,17,23
character(len=200):: fluxafile="Flux/flux.avec.bin"           !!  "
character(len=200):: spifile="Dat/xsec.spi.dat"               !! for action 16 (cross section)
character(len=200):: gtaufile="Dat/gtau.dat"                  !!  " (total flux(t) without e_ke resolution)
character(len=200):: projspifile="Dat/xsec.proj.spi"          !! for action 17 (partial cross section)
character(len=200):: projgtaufile="Dat/gtau.dat"              !!  " (projected flux(t))
character(len=200):: projfluxfile="Flux/proj.flux.wfn.bin"    !!  "
character (len=200):: catspffiles(50)="Bin/cation.spfs.bin"       !!  " (see numcatfiles in ACTIONS)
character (len=200):: catavectorfiles(50)="Bin/cation.avector.bin"!!  "
character(len=200):: angprojspifile="Dat/xsec.angproj.spi"    !!  " if angularflag.ne.0 calculate partial ionization, wrt angle
character(len=200):: fluxafile2="Flux/flux.avec.bin"          !! for action 23
character(len=200):: fluxmofile2="Flux/flux.mo.bin"           !!  "
character(len=200):: natplotbin="Bin/Natlorb.bin"             !! for actions 2,8
character(len=200):: spfplotbin="Bin/Spfplot.bin"             !! for actions 3,9
character(len=200):: denplotbin="Bin/Density.bin"             !! for actions 4,10
character(len=200):: rnatplotbin="Bin/RNatorb.bin"            !! for actions 5,11
character(len=200):: denprojplotbin="Bin/Denproj.bin"         !! for actions 6,12
character(len=200):: natprojplotbin="Bin/Natproj.bin"         !!  "
\end{verbatim} 
!!{\large \quad ACTIONS} \verb# may also be specified by Act=X where X is an integer on the command line #
\begin{verbatim}
integer :: numactions=0          !! 
integer :: actions(100)=0        !!              !! ACTIONS
!!   Act=1    Autocorrelation; set corrflag=1 for fourier transform
!!   Act=2    Save natorbs
!!   Act=3    Save spfs
!!   Act=4    Save density
!!   Act=5    Save R-natorbs
!!   Act=6    Save projections of natural configurations  (with Mathematica data in NatCurves/)
!!   Act=7    Save curve data files in LanCurves/ for Mathematica plotting.
!!   Act=8    Enter plotting mode (do not run calculation) and plot natorbs 
!!   Act=9    Enter plotting mode and plot spfs
!!   Act=10   Enter plotting mode and plot density
!!   Act=11   Enter plotting mode and plot natorbs in R
!!   Act=12   Enter plotting mode and plot projections from act=6
!!   Act=13   Nuclear FLUX 
!!   Act=14   Enter plotting mode and analyze nuclear flux
!!   Act=15   Save ELECTRONFLUX
!!   Act=16   Enter plotting mode and analyze ELECTRONFLUX 
!!   Act=17   Enter plotting mode and analyze ELECTRONFLUX (projected)
!!   Act=18   Plot denproj from act=6
!!   Act=19   Enforce natorbs between steps (experimental)
!!   Act=20   Overlaps with supplied eigenfunctions
!!   Act=21   Fourier transform dipole moment with pulse for emission/absorption
!!   Act=22   With Dirac Frenkel restrition, constraintflag=2, check norm of error - NEEDS
!!                 dfrestrictflag=2 not 1
!!   Act=23   Enter plotting/analysis mode and read flux.bin files from Act=15 for overlaps
!!                 between two time dependent wave functions
!!   Act=24   keprojector
!!   Act=25   make psistats.dat
!!   Act=26   mcscf_matel with supplied eigenfunctions 
\end{verbatim}
!!{\large \quad ACTION VARIABLES (also see filenames in INPUT/OUTPUT above)}
\begin{verbatim}
integer :: numovlfiles=1         !!  see ovlspffiles and ovlavectorfiles in INPUT/OUTPUT
integer :: nkeproj=200           !!  For keprojector ACTION 24
real*8 :: keprojminenergy=0.04d0 !!   "
real*8 :: keprojenergystep=0.04d0!!   "
real*8 :: keprojminrad=30        !!   "
real*8 :: keprojmaxrad=40        !!   "
real*8 :: eground=0.d0           !! Eground=     !! energy to shift fourier transform for ACTIONS 1 16 17
                                                 !!   (AUTOCORRELATION AND PHOTOIONIZATION)
complex*16 :: ceground=(0.d0,0d0)!!              !! input as complex-valued instead if you like
real*8 :: autotimestep=1.d0      !! ACTIONS 1 and 21 (autocorrelation and emission/absorption):
                                 !!   time step for fourier transform
\end{verbatim}
!!{\large \quad AUTOCORRELATION, PHOTOIONIZATION and EMISSION/ABSORPTION (actions 1,16,17,21)}
\begin{verbatim}
integer :: ftwindowlength=-99 !! FOR ACTIONS 1,16,17,21 options for damping function of time to be transformed
                              !!    if .ge.0, only damp at end (high frequency cutoff): last ftwindowlength 
                              !!    points in fourier transform are damped by cosine function
integer :: ftwindowpower=1    !! if ftwindowlength not set, use previous (v1.16) windowing function: multiply
                              !!    all points by cos(pi t / 2 / tmax)**ftwindowpower 
integer :: fttriwindow=1      !! If nonzero override other two options, do straight linear damping function
                              !!    NOW DEFAULT linear damping looks best v1.16
integer :: ftdiff=0           !! fourier transform derivative of dipole moment not dipole moment
\end{verbatim}
!!{\large \quad EMISSION/ABSORPTION (action 21)}
\begin{verbatim}
integer :: hanningflag=0         !! for hanning window set nonzero action 1 autocorr
integer :: diptime=100           !! For act=20, outputs copies every diptime atomic units
integer :: dipmodtime=100        !! do ft every autotimestep*dipmodtime
real*8 :: dipolesumstart=0d0,&   !! range for integration of oscillator strength (e.g. for sum rule), photon
     dipolesumend=0d0            !!    energy atomic units (Hartree) start and end, dipolesumend default set
                                 !!    based on &pulse namelist input (depending on omegas)
\end{verbatim}
!!{\large \quad PHOTOIONIZATION (actions 15,16,17)}
\begin{verbatim}
integer :: computeFlux=500, &    ! 0=All in memory other: MBs to allocate
     FluxInterval=50,&           !! Multiple of par_timestep at which to save flux
     FluxSkipMult=1              !! Read every this number of time points.  Step=FluxInterval*FluxSkipMult
integer :: nucfluxopt=1          !! Include imaginary part of hamiltonian from nuc ke 2=only that part
integer :: FluxOpType=1          !! 0=Full ham 1=halfnium 
integer :: numcatfiles=1         !! see catspffiles and catavectorfiles in INPUT/OUTPUT for action 17
integer :: angularflag=0         !! for action 17 calculate fully differential partial ionization, angular
                                 !!   output in angprojspifile (atom/diatom only)
!!$ IMPLEMENT ME (DEPRECATE fluxinterval as namelist input) 
!!$ real*8 :: fluxtimestep=0.1d0
\end{verbatim}
!!{\large \quad PLOTTING OPTIONS }
\begin{verbatim}
integer :: plotmodulus=10        !! PlotModulus= !! For saving nat/spf (Act=2, 6), par_timestep interval
real*8 :: plotpause=0.25d0       !! PlotPause=   !!    for saving natorbs.  plotskip is for stepping over
real*8 :: plotrange=0.2d0        !! PlotZ=       !!    the saved natorbs on read.  others are dimensions
real*8 :: plotcbrange=0.001d0    !!              !!    
real*8 :: plotxyrange=2.d0       !! PlotXY=      !!    
real*8 :: plotview1=70.d0        !!              !! viewing angle, degrees
real*8 :: plotview2=70.d0        !!              !! viewing angle, degrees
integer :: plotnum=10            !! PlotNum=     !! Max number of plots
integer :: plotterm=0            !!              !! 0=x11, 1=aqua
integer :: pm3d=1                !! PM3D         !! Turn pm3d on when plotting
integer :: plotres=50            !!              !! Resolution of plot
integer :: plotskip=1            !! PlotSkip=    !! For plotting (Act=3,5,7), number to skip over
real*8 :: povmult=1d0            !!  Mult df3 data by factor.  For small part of orbs.
integer :: povres=10             !!              !! Povray resolution
integer :: numpovranges=1        !!              !! number of magnifications to plot
real*8 :: povrange(10)=(/ 5,15,& !!              !! Povray plotting ranges (unitless - each magnification)
  80,80,80,80,80,80,80,80 /)
real*8  :: povsparse=1.d-3       !!              !! Sparsity threshold for transformation matrix in povray
\end{verbatim}
!!{\large \quad MISC AND EXPERIMENTAL}
\begin{verbatim}
integer :: pulsewindowtoo=1      !! use window function for E(omega) as well
logical :: readfullvector=.true.
logical :: walksinturn=.false.   !!              !! if you have problems with MPI i/o, maybe try this
integer :: turnbatchsize=5
integer :: nosparseforce=0       !!              !! to override exit with large number of configs, no sparse
integer :: noftflag=0            !!              !! turns off f.t. for flux. use for e.g. core hole propag'n.
integer :: timefacforce=0        !!              !!  override defaults
integer :: timedepexpect=0  !! expectation value of H_0(t) or H(t) reported
integer :: cmf_flag=1            !! CMF/VMF      !! CMF/LMF/QMF or VMF?
integer :: intopt=3              !! RK, GBS      !! SPF/VMF Integrator: 0, RK; 1, GBS, 2, DLSODPK  
                                                 !!  for CMF: 3=expo 4=verlet
integer :: verletnum=80          !!              !! Number of verlet steps per CMF step
integer :: jacprojorth=0         !! 1: projector = sum_i |phi_i> <phi_i|phi_i>^-1 <phi_i|
                                 !! 0:             sum_i |phi_i> <phi_i|    default
integer :: jacsymflag=0          !! 1:  use WP - PW  not (1-P)W   0: default (1-P)W
integer :: jacgmatthird=0        !! 0: default g (constraintflag.ne.0) is linear operator
                                 !! 1: g |phi_c> -> sum_ab |phi_a> g_ab <phi_b|phi_c>
real*8 :: autopermthresh=0.001d0 !! Autoperm=
real*8 :: autonormthresh=0.d0    !! 
integer :: debugflag=0
real*8 :: debugfac=1d0
\end{verbatim}
