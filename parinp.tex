\begin{verbatim}
module parameters
  use fileptrmod
  implicit none

!! recently added, otherwise notable. 

integer :: reinterp_orbflag=0                    !! sinc dvr only, half spacing interpolation for orb load
real*8 :: quadstarttime=-1d0
real*8 :: maxquadnorm=1d10                       !! brakes to use if improvedquadflag=2 or 3 is diverging
real*8 :: timestepfac=1d0                        !! accelerate relax
real*8 :: max_timestep=1d10                      !!

integer :: parorbsplit=1                         !!  Parallelize orbital calculation.  Might speed up, might
                                                 !!   slow down; check timing.
!! FOR TOTAL ORBITAL PARALLELIZATION with SINC DVR, SET PARORBSPLIT=3
!!   and orbparflag=.true. in &sinc_params.  parorbsplit=3 not supported for atom or diatom.

real*8 :: mshift=0d0                             !! shift configurations based on m-value.. to break 
                                                 !!  degeneracy for state averaged sym restricted
                                                 !!  (mrestrictmin, mrestrictmax) mcscf; good idea.
integer :: denmatfciflag=0                       !! If .ne. 0 then does denmat constrant as programmed
                                                 !!  before Miyagi's help

integer :: saveflag=1                            !! if zero does not save wave function

integer :: walkwriteflag=0                       !! Turning OFF writing of walks by default

!! *********************************************************************************************************** !!
!!   Parameters for MCTDHF calculation; parinp NAMELIST input from Input.Inp (default)
!! *********************************************************************************************************** !!
!!
!! Type, variable, default       !! Command      !! Description 
!!   value                       !!  line        !! 
                                 !!  option      !!
\end{verbatim}
!!{\large \quad MAIN PARAMETERS }
\begin{verbatim}
integer :: numelec=2             !!              !! NUMBER OF ELECTRONS
integer :: tdflag=0              !! Pulse        !! Use pulse?
integer :: mcscfnum=1            !! MCSCF=       !! Number of A-vectors (state avgd mcscf or prop)
integer :: sparseconfigflag=0    !! Sparse       !! Sparse configuration routines on or off (for large # configs)
integer :: orbcompact=0          !!              !! Compact orbitals for expo prop with spfrestrictflag?  Probably ok.
character (len=200) :: &         !!              !! MAY BE SET BY COMMAND LINE OPTION ONLY: not namelist
  inpfile="Input.Inp        "    !! Inp=filename !!  input.  (=name of input file where namelist input is)
\end{verbatim}
!!{\large \quad ORBITALS (SINGLE PARTICLE FUNCTIONS, SPFS)}
\begin{verbatim}
integer :: nspf=1                !! Nspf=        !! number of orbitals
integer :: numfrozen=0           !!              !! number of doubly occ orbs (removed from calculation)
integer :: spfrestrictflag=0     !!              !! Restrict m values of orbitals?  
integer :: spfmvals(1000)=0      !!              !!   M-values of orbitals 
integer :: spfugrestrict=0       !!              !! Restrict parity of orbitals? 
integer :: spfugvals(1000)=0     !!              !!   Parity (+/-1; 0=either) of orbitals (ungerade/gerade)
\end{verbatim}
!!{\large \quad CONFIGURATIONS}
\begin{verbatim}
integer :: mrestrictflag=0       !!              !! If spfrestrictflag=1, restrict wfn to given total M.
integer :: mrestrictval=0        !!              !!    This is the value.
integer :: mrestrictmax= 99999   !!              !! If doing state averaged MCSCF, can include a range of m vals;
integer :: mrestrictmin=-99999   !!              !!    set these variables, with mrestrictflag=0, spfrestrictflag=1
integer :: ugrestrictflag=0      !!              !! like mrestrictflag but for parity
integer :: ugrestrictval=1       !!              !!    like mrestrictval but for parity (1=even,-1=odd)
integer :: restrictflag=1        !!              !! Restrict spin projection of determinants?
integer :: restrictms=0          !!              !! For restrictflag=1: 2*m_s: 2x total m_s (multiplicity of 
                                                 !!    lowest included spin states minus one)
integer :: spinwalkflag=1        !!              !! Calculate spin info (required for below 2 options)
integer :: allspinproject=1      !!              !! Constrain S(S+1) for propagation?
integer :: spinrestrictval=0     !!              !! For allspinproject=1: determines spin. Default high spin S=M_s.
                                                 !!  To override use this variable. Equals 2S if S^2 eigval is S(S+1)
\end{verbatim}
!! {\large \quad For restricted configuration lists (not full CI): SEE MANUAL about dfrestrictflag}
\begin{verbatim}
integer :: numshells=1           !!              !! number of shells.  greater than one: possibly not full CI. 
!!integer :: shelltop(100)=-1    !! Numfrozen=   !! shelltop is namelist input in parinp; the internal variable 
                                                 !!    is allshelltop.   shelltop(1) only may be assigned via 
                                                 !!    command line, with Numfrozen.
integer :: numexcite(100)=99     !! Numexcite=   !! excitations from core shells (i.e. defined for shells 1 
                                                 !!    through numshells-1).  Only numexcite(1) may be 
                                                 !!    assigned via command line input.
integer :: minocc(100)=-999      !!              !! minimum occupation, each shell
integer :: maxocc(100)=999       !!              !!    maximum
integer :: vexcite=99            !!              !! excitations INTO last shell. Use to restrict to doubles, etc.
\end{verbatim}
!!{\large \quad INITIALIZATION}
\begin{verbatim}
integer :: loadspfflag=0         !! Spf=file     !! load spfs to start calculation?  
                                                 !!    (Otherwise, core eigenfunctions.)
integer :: numspffiles=1
integer :: numskiporbs=0         !!              !! Reading orbs on file(s), skips members of combined set.
integer :: orbskip(1000)=0       !!              !! Which to skip
integer :: loadavectorflag=0     !! A=file       !! load avector to start calculation?
integer :: numavectorfiles=1
character (len=200) :: &         !!              !! A-vector binary file to read.  Can have different configs
     avectorfile(100)="Bin/avector.bin"          !!   but should have same number of electrons.   
character (len=200) :: &         !!              !! Spf file to read.  Can have fewer m vals, smaller radial 
      spffile(100)="Bin/spfs.bin"!! Spf=filename !!   grid, or fewer than nspf total orbitals. 
integer :: avecloadskip(100)=0

integer :: numholes=0                            !! Load a-vector with this many more electrons and annihilate
integer :: numholecombo=1                        !! Number of (products of) annihilation operators to combine 
                                                 !!   (for spin adapt)
integer :: numloadfrozen=0                       !! For loading a vector with orbitals to be frozen (dangerous)
integer, allocatable :: myavectorhole(:,:,:)     !! Namelist input is avectorhole.  Fast index numholes (#
                                                 !!   annihilation operators to multiply together); then    
                                                 !!   numholecombo, number of such products to combine; then
                                                 !!   mcscfnum, wfn of current propagation.
integer :: excitations=0                         !! Similar to holes: number of products of excitation ops
integer :: excitecombos=1                        !!   number of products to linearly combine
integer,allocatable:: myavectorexcitefrom(:,:,:) !! Similar to avectorhole.  Namelist input avectorexcitefrom, etc.
integer,allocatable:: myavectorexciteto(:,:,:)   !!
                                                 !! For both excite and hole: value is spin orbital index
                                                 !! 1=1alpha, 2=1beta, 3=1alpha, etc.
                                                 !! negative input -> negative coefficent
\end{verbatim}
!!{\large \quad PROPAGATION/RELAXATION}
\begin{verbatim}
real*8 :: par_timestep=0.1d0     !! Step=        !! MEAN FIELD TIMESTEP
real*8 :: finaltime=4d4          !! T=           !! length of prop.  Overridden for pulse and relax.  
integer :: improvedrelaxflag=0   !! Relax        !! For improved versus regular relaxtion.   
integer :: threshflag=0          !!              !! Set to 1 for regular relaxation
integer :: improvedquadflag=0    !!              !! Use newton iteration not diagonalization for improvedrelax.
                                                 !!     (1 = A-vector, 2 = orbitals, 3 = both)
integer :: improvednatflag=0     !!              !! If improved relax, replace with natorbs every iter
                                                 !!    enforced & required for non full CI calculation
real*8 :: stopthresh=1d-5        !!              !! Spf error tolerance for relaxation convergence (PRIMARY)
real*8 :: astoptol=1d-7                          !! Avector error tolerance for relax  (BACKUP - WAS STOPTHRESH)
integer :: littlesteps=1         !!              !! Sub intervals of mean field time step for avector prop
integer :: maxexpodim=100        !!              !! Expokit maximum kry dimension
integer :: expodim=10            !!              !! Expokit starting krylov size
real*8  :: expostepfac=1.2d0     !!              !! 
real*8 :: expotol=1d-8           !!              !! Expokit krylov convergence 
real*8 :: denreg=1d-10           !! Denreg=      !! density matrix regularization parameter.
real*8 :: invtol=1d-12
\end{verbatim}
!!{\large \quad CONSTRAINT: Constraintflag. NEED FOR RESTRICTED CONFIG LIST.}
\begin{verbatim}
!!  1: Density matrix constraint: assume nothing, keep constant off block diag 
!!         (lioville solve)
!!  2: Dirac-Frenkel (McLachlan/Lagrangian) variational principle.

integer :: constraintflag=0      !! Constraint=  !! As described immediately above
real*8 :: lioreg= 1d-9           !!              !! Regularization for linear solve for both
integer :: dfrestrictflag=0      !!              !! apply constraint to configuration list?  Must use this
                                                 !!  option if constraintflag /= 0.  1 is sufficient;
                                                 !!  dfrestrictflag=2 necessary for action 22. 
                                                 !!  SEE MANUAL FOR PROPER USE OF dfrestrictflag/shell options.
integer :: conway=0                              !! for constraintflag=2, dirac frenkel constraint
                                                 !!   0=McLachlan 1=50/50 mix 2=Lagrangian
                                                 !!   3=Lagrangian with epsilon times McLachlan
real*8 :: conprop=1d-1                           !! epsilon for conway=3

\end{verbatim}
!!{\large \quad INPUT / OUTPUT }
\begin{verbatim}
character (len=200) ::      avectoroutfile="Bin/avector.bin"  !! A-vector output file.        
character (len=200) ::      spfoutfile="Bin/spfs.bin"         !! Spf output file.
character(len=200):: psistatsfile="Dat/psistats.dat"
character(len=200):: dendatfile="Dat/denmat.eigs.dat"
character(len=200):: denrotfile="Dat/denmat.rotate.dat"
character(len=200):: rdendatfile="Dat/rdenmat.eigs.dat"
character (len=200) :: ovlspffiles(50)="ovl.spfs.bin"
character (len=200) :: ovlavectorfiles(50)="ovl.avector.bin"
character(len=200):: zdipfile="Dat/ZDipoleexpect.Dat"
character(len=200):: zdftfile="Dat/ZDipoleft.Dat"
character(len=200):: ydipfile="Dat/YDipoleexpect.Dat"
character(len=200):: ydftfile="Dat/YDipoleft.Dat"
character(len=200):: xdipfile="Dat/XDipoleexpect.Dat"
character(len=200):: xdftfile="Dat/XDipoleft.Dat"
character(len=200):: corrdatfile="Dat/Correlation.Dat"
character(len=200):: corrftfile="Dat/Corrft.Dat"
character(len=200):: outovl="Dat/Overlaps.dat"
character(len=200):: fluxmofile="Flux/flux.mo.bin"
character(len=200):: fluxafile="Flux/flux.avec.bin"
character(len=200):: configlistfile="WALKS/configlist.BIN"
character(len=200):: fluxmofile2="Flux/flux.mo.bin"
character(len=200):: fluxafile2="Flux/flux.avec.bin"
character(len=200):: projfluxfile="Flux/proj.flux.wfn.bin"
character(len=200):: timingdir="timing"
character(len=200):: spifile="Dat/xsec.spi.dat"
character(len=200):: natplotbin="Bin/Natlorb.bin"
character(len=200):: spfplotbin="Bin/Spfplot.bin"
character(len=200):: denplotbin="Bin/Density.bin"
character(len=200):: denprojplotbin="Bin/Denproj.bin"
character(len=200):: natprojplotbin="Bin/Natproj.bin"
character(len=200):: rnatplotbin="Bin/RNatorb.bin"
\end{verbatim}
!!{\large \quad PULSE.  (If tdflag=1) }
\begin{verbatim}
integer :: numpulses=1
integer :: velflag=0             !!              !!  Length (V(t)) or velocity (A(t))       
integer ::  pulsetype(10)=1      !!              !!  Pulsetype=1:  A(t) = pulsestrength * sin(w t)^2,
real*8  :: omega(10)=1.d0        !!              !!  2:  A(t) = strength * sin(w t)^2 
real*8 :: omega2(10)=1.d0        !!              !!             * sin(w2 t + phaseshift),
real*8 :: pulsestart(10)=0.1d0   !!              !!   
real*8 :: phaseshift(10)=0.d0    !!              !!    pulsestart < t < pulsestart + pi/w; 0 otherwise
real*8 :: chirp(10)=0d0          !!              !!
real*8 :: ramp(10)=0d0
real*8 :: longstep(10)=1d0       !!              !!  Pulsetype 3 available: monochromatic, sinesq start+end
!! NOW COMPLEX
DATATYPE :: pulsestrength(10)=.5d0 !!            !!  A_0 = E_0/omega (strength of field)  
real*8 :: intensity(10)= -1.d0   !!              !! overrides pulse strength.  Intensity, 10^16 W cm^-2 
real*8 :: pulsetheta(10)=0.d0    !!              !!  angle between polarization and bond axis (radians)
real*8 :: pulsephi(10)=0.d0      !!              !!  polarization in xy plane
real*8 :: maxpulsetime=1.d20     !!              !!  
real*8 :: minpulsetime=0.d0      !!              !!  By default calc stops after pulse (overrides finaltime,
                                                 !!   numpropsteps); this will enforce minimum duration
 \end{verbatim}
!!{\large \quad Biorthogonalization }
\begin{verbatim}
integer ::      maxbiodim=100, &
     biodim=10                   !! Krylov dim for biorthogonalization
real*8 :: lntol=1d-12, &
     biotol=1.d-6
\end{verbatim} 
!!{\large \quad ACTIONS} \verb# may also be specified by Act=X where X is an integer on the command line #
\begin{verbatim}
integer :: numactions=0          !! 
integer :: actions(100)=0        !!              !! ACTIONS

!!   Act=1    Autocorrelation; set corrflag=1 for fourier transform
!!   Act=2    Save natorbs
!!   Act=3    Save spfs
!!   Act=4    Save density
!!   Act=5    Save R-natorbs
!!   Act=6    Save projections of natural configurations  (with Mathematica data in NatCurves/)
!!   Act=7    Save curve data files in LanCurves/ for Mathematica plotting.
!!   Act=8    Enter plotting mode (do not run calculation) and plot natorbs 
!!   Act=9    Enter plotting mode and plot spfs
!!   Act=10   Enter plotting mode and plot density
!!   Act=11   Enter plotting mode and plot natorbs in R
!!   Act=12   Enter plotting mode and plot projections from act=6
!!   Act=13   Nuclear FLUX 
!!   Act=14   Enter plotting mode and analyze nuclear flux
!!   Act=15   Save ELECTRONFLUX
!!   Act=16   Enter plotting mode and analyze ELECTRONFLUX 
!!   Act=17   Enter plotting mode and analyze ELECTRONFLUX (projected)
!!   Act=18   Plot denproj from act=6
!!   Act=19   Enforce natorbs between steps (experimental)
!!   Act=20   Overlaps with supplied eigenfunctions
!!   Act=21   Fourier transform dipole moment with pulse for emission/absorption
!!   Act=22   With Dirac Frenkel restrition, constraintflag=2, check norm of error - NEEDS
!!                 dfrestrictflag=2 not 1
!!   Act=23   Enter plotting/analysis mode and read flux.bin files from Act=15 for overlaps
!!                 between two time dependent wave functions
!!   Act=24   keprojector
integer :: nkeproj=200           !!  For keprojector
real*8 :: keprojminenergy=0.04d0 !!   "
real*8 :: keprojenergystep=0.04d0!!   "
real*8 :: keprojminrad=30        !!   "
real*8 :: keprojmaxrad=40        !!   "

integer :: hanningflag=0         !! for hanning window -- was corrflag
integer :: diptime=100           !! For act=20, outputs copies every diptime atomic units
integer :: dipmodtime=200        !! do ft every autotimestep*dipmodtime
integer :: numovlfiles=1
real*8 :: autopermthresh=0.001d0 !! Autoperm=
real*8 :: autonormthresh=0.d0    !! 
real*8 :: eground=0.d0           !! Eground=     !! energy to shift fourier transform 
complex*16 :: ceground=(0.d0,0d0)!!              !! input as complex-valued instead if you like
real*8 :: autotimestep=1.d0      !!
real*8 :: fluxtimestep=0.1d0     !!
integer :: nucfluxflag=0         !! 0 = both 1 =electronic 2= nuclear  NOT nuclear flux action 13,14
\end{verbatim}
!!{\large \quad PHOTOIONIZATION (actions 15,16,17)}
\begin{verbatim}
integer :: computeFlux=500, &      !! 0=All in memory other: MBs to allocate
     FluxInterval=50,&           !! Multiple of par_timestep at which to save flux
     nEFlux=1001,&               !! Number of energies in flux FT
     FluxSkipMult=1              !! Read every this number of time points.  Step=FluxInterval*FluxSkipMult
integer :: nucfluxopt=0          !! Include imaginary part of hamiltonian from nuc ke 
integer :: FluxSineOpt=1,&       !! Use windowng function
     FluxOpType=1                !! 0=Full ham 1=halfnium 
real*8 :: EFluxLo=0.01,&         !! Low energy boundary of F.T. (relative to eground)
     EFluxHi=2d0                 !! High energy boundary
integer :: FluxNBins=4           !! number of previous times to plot in xsec.spi.dat
\end{verbatim}
!!{\large \quad PLOTTING OPTIONS }
\begin{verbatim}
integer :: plotmodulus=10        !! PlotModulus= !! For saving nat/spf (Act=2, 6), par_timestep interval
real*8 :: plotpause=0.25d0       !! PlotPause=   !!    for saving natorbs.  plotskip is for stepping over
real*8 :: plotrange=0.2d0        !! PlotZ=       !!    the saved natorbs on read.  others are dimensions
real*8 :: plotcbrange=0.001d0    !!              !!    
real*8 :: plotxyrange=2.d0       !! PlotXY=      !!    
real*8 :: plotview1=70.d0        !!              !! viewing angle, degrees
real*8 :: plotview2=70.d0        !!              !! viewing angle, degrees
integer :: plotnum=10            !! PlotNum=     !! Max number of plots
integer :: plotterm=0            !!              !! 0=x11, 1=aqua
integer :: pm3d=1                !! PM3D         !! Turn pm3d on when plotting
integer :: plotres=50            !!              !! Resolution of plot
integer :: plotskip=1            !! PlotSkip=    !! For plotting (Act=3,5,7), number to skip over
real*8 :: povmult=1d0            !!  Mult df3 data by factor.  For small part of orbs.
integer :: povres=10             !!              !! Povray resolution
integer :: numpovranges=1        !!              !! number of magnifications to plot
real*8 :: povrange(10)=(/ 5,15,& !!              !! Povray plotting ranges (unitless - each magnification)
  80,80,80,80,80,80,80,80 /)
real*8  :: povsparse=1.d-3       !!              !! Sparsity threshold for transformation matrix in povray
\end{verbatim}
!!{\large \quad SPARSE - if sparseconfigflag .ne. 0}
\begin{verbatim}
integer :: sparseopt =1                          !! 0= direct CI  1= sparse matrix algebra (faster, more memory)
integer :: nosparseforce=0       !!              !! to override exit with large number of configs, no sparse
integer :: maxaorder=1000        !!              !!   lanczos order for sparse a-vector prop and improvedquad=2
integer :: aorder=30             !!              !!   
integer :: lanprintflag=0
integer :: lanczosorder=200      !!              !!   lanczos order used in A-vector eigen.
integer :: lancheckstep=20       !!              !! lanczos eigen routine checks for convergence every this # steps
real*8 :: aerror=1d-9            !!              !! lanczos error criterion for sparse a-vector CMF propagation
                                                 !!  within aerror to stop.
!!real*8 :: lanthresh=1.d-7
real*8 :: lanthresh=1.d-9       !!              !! convergence criterion.
\end{verbatim}
!!{\large \quad MISC AND EXPERIMENTAL}
\begin{verbatim}
integer :: iprintconfiglist=0
integer :: drivingflag=0                         !!  Solve for the change in the wave function not wave function 
real*8 :: drivingproportion=0.999999999999d0     !!   -- "psi-prime" treatment.
real*8 :: quadtol=1d-1           !!              !! Threshold for solution of Newton solve iterations.  For
                                                 !!    difficult cases increase this value.
integer :: quadprecon=1          !!              !! Precondition newton iterations
real*8 :: quadpreconshift=0d0
real*8 :: quadthresh=1.d+8       !!              !!   Threshold for turning on quadratic convergence.
integer :: spf_flag=1            !!              !! IF ZERO, FREEZE SPFS. (for debugging, or TDCI)
integer :: avector_flag=1        !!              !! IF ZERO, FREEZE AVECTOR. (for debugging)
integer :: noftflag=0            !!              !! turns off f.t. for flux. use for e.g. core hole propag'n.
integer :: timefacforce=0        !!              !!  override defaults
DATATYPE :: timefac=&            !! Prop/        !! d/dt psi = timefac * H * psi
        DATANEGONE               !!  Relax       !!
integer :: timedepexpect=0
integer ::  checktdflag=0   !!  makes pulse constant and evaluates expectation value of h(t) 
                            !!  to check energy conservation for dipole operators (consistency 
                            !!  between reduced and matel)
integer :: dipolewindowpower=1   !!  multiply by cosine^dipolewindowpower for dipole ft
integer :: diffdipoleflag=1 !! fourier transform derivative of dipole moment not dipole moment

integer :: cmf_flag=1            !! CMF/VMF      !! CMF/LMF/QMF or VMF?
integer :: intopt=3              !! RK, GBS      !! SPF/VMF Integrator: 0, RK; 1, GBS, 2, DLSODPK  
                                                 !!  for CMF: 3=expo 4=verlet
integer :: verletnum=80          !!              !! Number of verlet steps per CMF step
integer :: jacprojorth=0         !! 1: projector = sum_i |phi_i> <phi_i|phi_i>^-1 <phi_i|
                                 !! 0:             sum_i |phi_i> <phi_i|
integer :: jacunitflag=0         !! 1:  (1 x v)_j = sum_i |v_i><v_i|v_j>  for homogeneous third order
                                 !! 0: usual          
integer :: jacsymflag=0          !! 3:  use 1-PHP  not (1-P)H
integer :: biocomplex=0          !! 1=old way complex zg/hpiv  0=always real

integer :: debugflag=0
real*8 :: debugfac=1d0

integer :: nonsparsepropmode=1   !! 0 = ZGCHBV expokit; 1 = mine expmat
\end{verbatim}
