\begin{verbatim}

!! PARAMETERS FILE FOR 1-D SINC DVR BASIS AND HAMILTONIAN:
!!            sinc1dparinp namelist input.

#include "Definitions.INC"

module myparams
implicit none

!! HAMILTONIAN PARAMS

real*8  :: twostrength=1d0   !!  2-electron interaction coef.
real*8  :: nucstrength=1d0   !!  2-nuclei interaction coef.
                             !!  Two-particle interaction:
integer :: twotype = 1       !!  1 = potential interaction  0 = constant interaction
integer :: twomode = 0       !!  If potential: 0 = sech^2 potential  1 = soft coulomb

integer :: numcenters=1
integer :: centershift(100)=0       !! grid point index for each center
real*8 :: nuccharges(100)=2d0       !! nuclear charges
real*8 :: softness=0.5d0            !! scale parameter for sech and softening param for coulomb
                                    !!  softcoul = 1/sqrt(x^2+softness^2)

!! BASIS PARAMS

integer :: numpoints=15
real*8 :: spacing=0.25d0

DATATYPE :: harmstrength=0d0   !! harmonic potential

!!    CAP & Exterior Complex Scaling

integer :: capflag=0               !! Number of complex absorbing potentials
integer :: capmode=0               !! Capmode=1 is 
integer :: cappower(100)=2         !!   v_i(r)= capstrength_i*(r/capstart_i)^cappower_i
real*8 :: capstart(100)=0.001d0    !! Capmode=0 is    
real*8 :: capstrength(100)=0.01d0  !!   v_i(r)= capstrength_i*max(0,r-capstart_i)^cappower_i
real*8 :: mincap=0d0 , maxcap=1d30 !! V_CAP = -i* max(mincap,min(maxcap,sum_i v_i))

integer :: scalingflag=0           !! 1 = SMOOTH EXTERIOR COMPLEX SCALING
real*8 :: scalingdistance=10000d0  !! atomic units (bohr)
real*8 :: smoothness=5             !! atomic units (bohr)
real*8 :: scalingtheta=0d0         !! scaling angle
real*8 :: scalingstretch=1d0         !! stretching factor

!! NUMERICAL PARAMS

integer :: eigmode = 0       !!  0 = exact lapack diag 1 = block lanczos

!!    FOR KE MULT IN 1D, toepflag.ne.0 does single toeplitz matvec
integer :: toepflag=0

!!    FOR TOTAL ORBITAL PARALLELIZATION, SET orbparflag=.true., AND parorbsplit=3 in &parinp
logical :: orbparflag=.false.

!!    THE FOLLOWING FLAG IS THEN RELEVANT.  Option for parallel KE matvec, rate limiting step.
integer :: zke_paropt=1   !! 0=sendrecv 1=SUMMA (bcast before) 2=reduce after

!!    fft_batchdim: determines batch size for matrix elements and 
!!    fft_circbatchdim: determines sub batch size for FFT 
!!       defaults set small (less memory, more MPI messages) to avoid MPI problems when doing large 
!!       calculations.  Otherwise bigger values will be faster.  There is a message size sweet spot
!!       on many machines.
integer :: fft_batchdim=1     !! 1 = do nspf matrix elements in nspf batches (less memory)
                              !! 2 = do nspf^2 in one batch (faster unless MPI problems)
integer :: fft_circbatchdim=1 !! 0,1,2, circbatchdim < batchdim; larger faster unless MPI problems
integer :: fft_ct_paropt=1           !! fft_ct_paropt,
                                     !!    like zke_paropt: 0 = sendrecv 1 = summa
integer :: num_skip_orbs=0
integer :: orb_skip(200)=-1

!!  for block lanzos:

integer :: orblanorder=500         !! krylov order for block lanczos orbital calculation
integer :: orblancheckmod=10       !! check every
real*8 :: orblanthresh=1d-4

!! construct virtual orbtials using density based on loaded orbitals
integer :: ivoflag=0
real*8 :: loadedocc(200)=2d0
\end{verbatim}
